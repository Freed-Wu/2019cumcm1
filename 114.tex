% Fakesection 序言之前

\RequirePackage[l2tabu, orthodox]{nag}
\RequirePackage{ifxetex}
\RequireXeTeX

\documentclass{ctexart}

%颜色
\usepackage{xcolor}

%长度
\usepackage{printlen}
\uselengthunit{mm}

%图形
\usepackage{pifont}
\usepackage{ean13isbn}
\usepackage{qrcode}
\usepackage{pdfpages}
\usepackage{overpic}
\usepackage{graphicx}
\graphicspath{{./src/}}
\usepackage{media9}
\usepackage{wallpaper}
\usepackage{wrapfig}

%表格
\usepackage{tabu}
\usepackage{longtable}
\usepackage{booktabs}
\usepackage{diagbox}
\usepackage{multicol}
\usepackage{multirow}
\usepackage{makecell}
\usepackage{fancybox}
\usepackage{colortbl}
\usepackage{tcolorbox}
\tcbuselibrary{skins}
\tcbuselibrary{breakable}
\tcbuselibrary{theorems}
\tcbuselibrary{listings}
\tcbuselibrary{xparse}
\tcbuselibrary{minted}% 用minted排版代码
\usepackage{fvextra}
\usepackage{csvsimple}
\usepackage{boxedminipage2e}

%公式
\usepackage{amsmath}
\usepackage{amsthm}
\usepackage{amsfonts}
\usepackage{amssymb}
\usepackage{amsbsy}
\usepackage{amsopn}
\usepackage{amstext}
\usepackage{mathrsfs}
\usepackage{bm}
\usepackage{textcomp}
\usepackage{latexsym}
\usepackage{exscale}
\usepackage{relsize}
%\usepackage{xymtex}
\usepackage{physics}
\usepackage{siunitx}
\usepackage{hologo}
\usepackage{cases}

%文字
\usepackage{csquotes}
\usepackage{microtype}

%正文
\usepackage{fancyhdr}
\usepackage{geometry}
\usepackage{lastpage}
\usepackage{indentfirst}
\usepackage{setspace}
\renewcommand\arraystretch{1.5}

%非正文
\usepackage{makeidx}
\makeindex
\usepackage{epigraph}
\usepackage{varwidth}
\usepackage{exercise}
\usepackage{tasks}

%参考文献
\usepackage{morewrites}
\renewcommand{\thefootnote}{\fnsymbol{footnote}}
\usepackage[resetlabels]{multibib}
\usepackage[numbers,sort&compress]{natbib}

%标题
\usepackage{caption}
\usepackage{subcaption}
\DeclareCaptionLabelFormat{andtable}%
{#1#2~\&~\tablename\thetable}
\newcounter{sub}

%其它
\usepackage{atbegshi}
\usepackage{lipsum}

\csname
endofdump
\endcsname

%代码
\usepackage{minted}

%链接%与beamer 冲突
\usepackage
[	colorlinks = true,
linkcolor = gray,
citecolor = gray,
backref=page
]{hyperref}

%枚举%与beamer 干涉
\usepackage{enumitem}
\setlist[enumerate, 2]
{	fullwidth,
	label = \alph*.,
	font = \textup,
	itemindent=2em
}

%标题%与beamer 冲突
\usepackage{titlesec}
%\titleformat{\chapter}{\centering\Huge\bfseries}{实验\chinese{chapter}~}{0pt}{}
\titleformat{\section}{\centering\LARGE\bfseries}{\S\ifthenelse{\value{section}=0}{}{\thesection}~}{0pt}{}
%\titleformat{\subsection}{\Large}{\chinese{subsection}、~}{0pt}{}
%\titleformat{\subsubsection}{\large}{\arabic{subsubsection}.~}{0pt}{}

\begin{document}

% Fakesection 摘要

\title{\textbf{乘坐高铁还是传统火车的行为分析}}
\author{}
\date{}
\maketitle

\renewcommand{\abstractname}{\Large 摘要}
\begin{abstract}
	本文
\end{abstract}

\textbf{关键词:;;。}

\newpage

% Fakesection 目录

\pagenumbering{roman}

\tableofcontents

\newpage

\listoffigures

\newpage

\listoftables

\newpage

\setcounter{section}{-1}

\pagenumbering{arabic}

\section{引言}%
\label{sec:引言}

\subsection{问题背景}%
\label{sub:问题背景}

\subsection{问题信息}%
\label{sub:问题信息}

\subsection{问题重述}%
\label{sub:问题重述}



\begin{enumerate}
	\item ;
	\item 。
\end{enumerate}

\newpage

\section{问题分析}%
\label{sec:问题分析}

\subsection{假设}%
\label{sub:假设}

\subsection{标记}%
\label{sub:标记}

\begin{table}[htpb]
	\centering
	\caption{标记}
	\label{tab:标记}
	\begin{tabu}to\linewidth{@{}X[c]X[c]@{}}
		\toprule
		变量&定义 \\
		\midrule
			& \\
			\bottomrule
	\end{tabu}

\end{table}

\subsection{分析}%
\label{sub:分析}

\subsubsection{题一}%
\label{ssub:题一}

\newpage

\section{模型1}%
\label{sec:模型1}

\newpage

\section{模型2}%
\label{sec:模型2}

\newpage

\section{模型3}%
\label{sec:模型3}

\newpage

\section{模型的评价与改进}%
\label{sec:模型的评价与改进}

\subsection{优点}%
\label{sub:优点}

\subsection{缺点}%
\label{sub:缺点}

\subsection{改进}%
\label{sub:改进}

\subsection{推广}%
\label{sub:推广}

\subsection{展望}%
\label{sub:展望}

\newpage

% Fakesection 参考文献

\bibliographystyle{IEEEtran}
%\bibliography{src/main}

% Fakesection 附录

\renewcommand{\thesection}{\Alph{section}~}

\appendix

\section{数据}%
\label{sec:数据}

\section{代码}%
\label{sec:代码}

% Fakesection 索引

\printindex

\end{document}

