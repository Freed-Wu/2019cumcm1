\RequirePackage[l2tabu, orthodox]{nag}
\RequirePackage{ifxetex}
\RequireXeTeX

\documentclass[tikz]{standalone}

%颜色
\usepackage{xcolor}

%长度
\usepackage{printlen}
\uselengthunit{mm}

%图形
\usepackage{pifont}
\usepackage{ean13isbn}
\usepackage{qrcode}
\usepackage{pdfpages}
\usepackage{overpic}
\usepackage{graphicx}

\usepackage{pstricks}

\usepackage{pgfplots}
\pgfplotsset{compat=1.16}
\usepackage{pgfmath}
\usepackage{pgf-pie}
\usetikzlibrary{calc}
\usetikzlibrary{shapes.geometric}
\usetikzlibrary{patterns}
\usetikzlibrary{arrows}
\usetikzlibrary{shapes}
\usetikzlibrary{chains}
\usetikzlibrary{mindmap}
\usetikzlibrary{graphs}
\usetikzlibrary{decorations.text}
\usetikzlibrary{arrows.meta}
\usetikzlibrary{shadows.blur}
\usetikzlibrary{shadings}

\usepackage{scsnowman}
\usepackage{circuitikz}
\usepackage{tikzpeople}
\usepackage{tikzducks}
\usepackage{flowchart}
\usepackage{smartdiagram}
\usepackage[edges]{forest}

%公式
\usepackage{amsmath}
\usepackage{amsthm}
\usepackage{amsfonts}
\usepackage{amssymb}
\usepackage{amsbsy}
\usepackage{amsopn}
\usepackage{amstext}
\usepackage{mathrsfs}
\usepackage{bm}
\usepackage{textcomp}
\usepackage{latexsym}
\usepackage{exscale}
\usepackage{relsize}
%\usepackage{xymtex}
\usepackage{physics}
\usepackage{siunitx}
\usepackage{hologo}
\usepackage{cases}

%文字
\usepackage{csquotes}
\usepackage{microtype}
\usepackage{ctex}

\csname
endofdump
\endcsname

\begin{document}

\begin{tikzpicture}[scale=1, transform shape]
	\tikzset{node distance=1.5cm}
	\node (开始) [draw, align=center, terminal] {开始};
	\node (输入数据) [draw, align=center, trapezium, trapezium left angle=70, trapezium right angle=110, below of= 开始] {输入数据};
	\node (聚类分析) [draw, align=center, process, below of= 输入数据] {聚类分析};
	\node (得到距离较近的因子) [draw, align=center, trapezium, trapezium left angle=70, trapezium right angle=110, below of= 聚类分析] {得到距离较近的因子};
	\node (对一次项和因子相乘得到的交叉项做最小二乘回归分析) [draw, align=center, process, below of= 得到距离较近的因子] {对一次项和因子相乘\\得到的交叉项做\\最小二乘回归分析};
	\node (利用回归系数进行预测) [draw, align=center, process, below of= 对一次项和因子相乘得到的交叉项做最小二乘回归分析] {利用回归系数进行预测};
	\node (结束) [draw, align=center, terminal, below of= 利用回归系数进行预测] {结束};
	\draw [-Stealth](开始) -- (输入数据);
	\draw [-Stealth](输入数据) -- (聚类分析);
	\draw [-Stealth](聚类分析) -- (得到距离较近的因子);
	\draw [-Stealth](得到距离较近的因子) -- (对一次项和因子相乘得到的交叉项做最小二乘回归分析);
	\draw [-Stealth](对一次项和因子相乘得到的交叉项做最小二乘回归分析) -- (利用回归系数进行预测);
	\draw [-Stealth](利用回归系数进行预测) -- (结束);
\end{tikzpicture}

\end{document}

